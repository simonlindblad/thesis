\texttt{Supervised machine learning models require a high quality set of labeled data in order to perform well.
This set of data is sometimes easy to obtain, but when it comes to text that is usually not the case.
Available text data often exists in abundance, but it is seldom labeled.
Labeling text data is a time consuming process, especially in the case where multiple labels can be assigned to a single text document.}

\texttt{The purpose of this thesis work was to make the labeling process of clinical reports as effective and effortless as possible by evaluating different multi-label active learning techniques.
By using one of these techniques the goal was to reduce the number of labeled documents needed, and increase the quality of those that is.
Binary Version Space Minimization, Maximum Loss Reduction with Maximum Confidence and Adaptive Active Learning are the three methods evaluated.}

\texttt{The Reuters dataset was used for comparisons and simulation of the labeling process.
Using Binary Version Space Minimization, an accuracy of 89\% was achieved with 2500 reports, compared to 85\% with random sampling.
With Adaptive Active Learning, 85\% accuracy could be reached after labeling 975 reports, compared to 1700 reports with random sampling.}

\texttt{The focus was on clinical reports, and a part of this study was to filter out the reports describing examinations that did not occur.
This was done by using an LDA model, where a couple of topics were able to identify reports describing canceled examinations.}