%%% lorem.tex --- 
%% 
%% Filename: lorem.tex
%% Description: 
%% Author: Ola Leifler
%% Maintainer: 
%% Created: Wed Nov 10 09:59:23 2010 (CET)
%% Version: $Id$
%% Version: 
%% Last-Updated: Wed Nov 10 09:59:47 2010 (CET)
%%           By: Ola Leifler
%%     Update #: 2
%% URL: 
%% Keywords: 
%% Compatibility: 
%% 
%%%%%%%%%%%%%%%%%%%%%%%%%%%%%%%%%%%%%%%%%%%%%%%%%%%%%%%%%%%%%%%%%%%%%%
%% 
%%% Commentary: 
%% 
%% 
%% 
%%%%%%%%%%%%%%%%%%%%%%%%%%%%%%%%%%%%%%%%%%%%%%%%%%%%%%%%%%%%%%%%%%%%%%
%% 
%%% Change log:
%% 
%% 
%% RCS $Log$
%%%%%%%%%%%%%%%%%%%%%%%%%%%%%%%%%%%%%%%%%%%%%%%%%%%%%%%%%%%%%%%%%%%%%%
%% 
%%% Code:

\chapter{Results}
\label{cha:results}

This chapter presents the results. Note that the results are presented
factually, striving for objectivity as far as possible.  The results
shall not be analyzed, discussed or evaluated.  This is left for the
discussion chapter.

In case the method chapter has been divided into subheadings such as
pre-study, implementation and evaluation, the result chapter should
have the same sub-headings. This gives a clear structure and makes the
chapter easier to write.

In case results are presented from a process (e.g. an implementation
process), the main decisions made during the process must be clearly
presented and justified. Normally, alternative attempts, etc, have
already been described in the theory chapter, making it possible to
refer to it as part of the justification.

%%%%%%%%%%%%%%%%%%%%%%%%%%%%%%%%%%%%%%%%%%%%%%%%%%%%%%%%%%%%%%%%%%%%%%
%%% lorem.tex ends here

%%% Local Variables: 
%%% mode: latex
%%% TeX-master: "demothesis"
%%% End: 

\section{EXPERIMENT 1}


The distribution of the most likely topics for the invalid reports can be seen in [FIGURE].
As can be seen, topic 17 and topic 34 were the most likely.
They were combined by ..... (SPECIFICS FOR THE CHOSEN MODEL).

The distribution over the invalid reports by cluster can be seen in [FIGURE].
The selected cluster was 25 based on the overwhelming majority.

\section{EXPERIMIENT 2}


Since there was a pattern some active learning approaches using different forms of clustering, such as Dasgupta et al\@'s approach using hierarchical clustering~\cite{dasgupta2008hierarchical}.
However, it is made for the single-label case with no obvious way of extending the technique into multi-label.
The same applies to the density based technique suggested by Attenberg et al\@.~\cite{attenberg2013class}.
