\chapter{Results}
\label{cha:results}

%This chapter presents the results. Note that the results are presented
%factually, striving for objectivity as far as possible.  The results
%shall not be analyzed, discussed or evaluated.  This is left for the
%discussion chapter.

%In case the method chapter has been divided into subheadings such as
%pre-study, implementation and evaluation, the result chapter should
%have the same sub-headings. This gives a clear structure and makes the
%chapter easier to write.

%In case results are presented from a process (e.g. an implementation
%process), the main decisions made during the process must be clearly
%presented and justified. Normally, alternative attempts, etc, have
%already been described in the theory chapter, making it possible to
%refer to it as part of the justification.

In this chapter the results are described.
First, the outcome from the exploratory study is presented, followed by the different experiments.
The first experiment, filtering out invalid reports, presents the evaluation of the topic model used to filter out the reports, as well as the specific topics and how they were used in the process.
In the second one, the methods considered and the decisions behind which ones that were appropriate are presented.
Finally, the last section goes through the result of evaluating the different active learning techniques.

\section{Exploratory Study}


\section{Filter Out Invalid Clinical Reports Using Topic Models and Clustering}\label{sec:exp1-result}




\section{Alternatives to Labeling at Random}



