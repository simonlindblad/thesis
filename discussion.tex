\chapter{Discussion}
\label{cha:discussion}

%This chapter contains the following sub-headings.

\section{Results}
\label{sec:discussion-results}

%Are there anything in the results that stand out and need be
%analyzed and commented on? How do the results relate to the
%material covered in the theory chapter? What does the theory
%imply about the meaning of the results? For example, what
%does it mean that a certain system got a certain numeric value
%in a usability evaluation; how good or bad is it? Is there
%something in the results that is unexpected based on the
%literature review, or is everything as one would theoretically
%expect?

\section{Method}
\label{sec:discussion-method}

Results that are based on a public dataset is naturally easier to reproduce.
The method becomes more reliable in the sense that the same results can be expected by reproducing the concrete steps.
However, the clinical dataset from Sectra is not publically available.
Therefore any results that are derived from specific attributes of that dataset might not be reasonable to expect when applied in a new environment.

The comparisons of the active learning algorithms are using the Reuters dataset, which both public and a standard dataset for evaluation.
Reproducing these results might therefore be an easier tasks.
Any issues with the reliability in this case might come from the implementation of the presented in Chapter~\ref{cha:theory}.
The implementations of the Binary Version Space Minimization, Maximum Loss Reduction with Maximum Confidence and Adaptive Active Learning are based on the algorithms, but written independently for this thesis.
They have not faced any public scrutiny and their correctnes can not be guaranteed.

The usage and analysis of the LDA model was done in a rather way.
During the exploration phase and the first experiment, the relationships and pattern found was in part a result of the authors intuition.
For this reason, if another party would perform the same study they might come up with different results and find other patterns.
However, the relationships found between the topics and the invalid reports are rather explicit, so the main difference between different researchers might be the specific thresholds.
For example, the 10\% threshold for prominent topics was chosen based on the author's experience with the dataset after seeing several iterations of topic models.
By using logistic regression for comparisson the subjectiveness of the thresholds can be compared with an objective baseline, which improves the reliability of the results.
Any subsequent study is more likely to obtain similar results with this approach.

Another aspect that is questionable when it comes to the first experiment is the labeling of the invalid reports.
This was done manually by the author, without any medical knowledge.
However, the nature of the labeling is rather trivial.
The medical knowledge required to understand the result of an examination is far greater than the one needed to see if an examination was performed.
Which in most cases can be identified not despite of, but because of the lack of medical information.
The number of labeled reports, X, is probably sufficient to manually identify broad relationships to set some categorization rules based on the output of the LDA.
However, the logistic regression based classification might have been able to achieve better results with more data to train on.
This could have been done rather easily without the time contraints of this thesis work.
In order to obtain a more balanced sample to train on, an active learning system could, as we have seen, be used.
However, the system developed in this thesis was not finished when the labeling was done, and it is not focused on the binary classification task.
Evaluating the categorization of invalid reports by accuracy, precision, recall and F1-score are fairly standard.
The metrics have been used in a lot of text classification and information retrieval research~\cite{aggarwal2012surveyclass, bishop2006pattern}.

%GThis is where the applied method is discussed and criticized.
%GTaking a self-critical stance to the method used is an
%Gimportant part of the scientific approach.
%G
%GA study is rarely perfect. There are almost always things one
%Gcould have done differently if the study could be repeated or
%Gwith extra resources. Go through the most important
%Glimitations with your method and discuss potential
%Gconsequences for the results. Connect back to the method
%Gtheory presented in the theory chapter. Refer explicitly to
%Grelevant sources.
%G
%GThe discussion shall also demonstrate an awareness of methodological
%Gconcepts such as replicability, reliability, and validity. The concept
%Gof replicability has already been discussed in the Method chapter
%G(\ref{cha:method}). Reliability is a term for whether one can expect
%Gto get the same results if a study is repeated with the same method. A
%Gstudy with a high degree of reliability has a large probability of
%Gleading to similar results if repeated. The concept of validity is,
%Gsomewhat simplified, concerned with whether a performed measurement
%Gactually measures what one thinks is being measured. A study with a
%Ghigh degree of validity thus has a high level of credibility. A
%Gdiscussion of these concepts must be transferred to the actual context
%Gof the study.
%G
%GThe method discussion shall also contain a paragraph of
%Gsource criticism. This is where the authors’ point of view on
%Gthe use and selection of sources is described.
%G
%GIn certain contexts it may be the case that the most relevant
%Ginformation for the study is not to be found in scientific
%Gliterature but rather with individual software developers and
%Gopen source projects. It must then be clearly stated that
%Gefforts have been made to gain access to this information,
%Ge.g. by direct communication with developers and/or through
%Gdiscussion forums, etc. Efforts must also be made to indicate
%Gthe lack of relevant research literature. The precise manner
%Gof such investigations must be clearly specified in a method
%Gsection. The paragraph on source criticism must critically
%Gdiscuss these approaches.
%G
%GUsually however, there are always relevant related research.
%GIf not about the actual research questions, there is certainly
%Gimportant information about the domain under study.
%G
%G\section{The work in a wider context}
%G\label{sec:work-wider-context}
%G
%GThere must be a section discussing ethical and societal
%Gaspects related to the work. This is important for the authors
%Gto demonstrate a professional maturity and also for achieving
%Gthe education goals. If the work, for some reason, completely
%Glacks a connection to ethical or societal aspects this must be
%Gexplicitly stated and justified in the section Delimitations in
%Gthe introduction chapter.
%G
%GIn the discussion chapter, one must explicitly refer to sources
%Grelevant to the discussion.
%G
%G%%%%%%%%%%%%%%%%%%%%%%%%%%%%%%%%%%%%%%%%%%%%%%%%%%%%%%%%%%%%%%%%%%%%%%
%G%%% lorem.tex ends here
%G
%G%%% Local Variables: 
%G%%% mode: latex
%G%%% TeX-master: "demothesis"
%G%%% End: 
%G
